\chapter{Requirements\label{ref-005}}

[This section shall be divided into the following paragraphs to specify the software item requirements, that is, those characteristics of the software item that are conditions for its acceptance. Software item requirements are software requirements generated to satisfy the system requirements allocated to this software item. Each requirement shall be assigned a project-unique identifier to support testing and traceability and shall be stated in such a way that an objective test can be defined for it. Each requirement shall be annotated with associated qualification method(s) (see Section 4) and traceability to system (or subsystem, if applicable) requirements (see Section 5, first bullet) if not provided in those sections. The degree of detail to be provided shall be guided by the following rule: Include those characteristics of the software item that are conditions for software item acceptance; defer to design descriptions those characteristics that the acquirer is willing to leave up to the developer. If there are no requirements in a given paragraph, the paragraph shall so state. If a given requirement fits into more than one paragraph, it may be stated once and referenced from the other paragraphs.]

\section{Required States and Modes\label{ref-006}}

[If the software item is required to operate in more than one state or mode having requirements distinct from other states or modes, this paragraph shall identify and define each state and mode. Examples of states and modes include: idle, ready, active, post-use analysis, training, degraded, emergency, backup, wartime, and peacetime. The distinction between states and modes is arbitrary. A software item may be described in terms of states only, modes only, states within modes, modes within states, or any other scheme that is useful. If no states or modes are required, this paragraph shall so state, without the need to create artificial distinctions. If states and/or modes are required, each requirement or group of requirements in this specification shall be correlated to the states and modes. The correlation may be indicated by a table or other method in this paragraph, in an appendix referenced from this paragraph, or by annotation of the requirements in the paragraphs where they appear.]

\section{Software Item Capability Requirements\label{ref-007}}

[This paragraph shall be divided into subparagraphs to itemize the requirements associated with each capability of the software item. A ``capability'' is defined as a group of related requirements. The word ``capability'' may be replaced with ``function,'' ``subject,'' ``object,'' or other term useful for presenting the requirements.]

\subsection{[Software Item Capability]\label{ref-008}}

[This paragraph shall identify a required software item capability and shall itemize the requirements associated with the capability. If the capability can be more clearly specified by dividing it into constituent capabilities, the constituent capabilities shall be specified in subparagraphs. The requirements shall specify required behavior of the software item and shall include applicable parameters, such as response times, throughput times, other timing constraints, sequencing, accuracy, capacities (how much/how many), priorities, continuous operation requirements, and allowable deviations based on operating conditions. The requirements shall include, as applicable, required behavior under unexpected, disallowed, or ``out of bounds'' conditions, requirements for error handling, and any provisions to be incorporated into the software item to provide continuity of operations in the event of emergencies. Paragraph 3.3.X of this document provides a list of topics to be considered when specifying requirements regarding inputs the software item must accept and outputs it must produce.]

\section{Software Item External Interface Requirements\label{ref-009}}

[This paragraph shall be divided into subparagraphs to specify the requirements, if any, for the software item's external interfaces. This paragraph may reference one or more Interface Requirements Specifications (IRSs) or other documents containing these requirements.]

\subsection{Interface Identification and Diagrams\label{ref-010}}

[This paragraph shall identify the required external interfaces of the software item (that is, relationships with other entities that involve sharing, providing, or exchanging data). The identification of each interface shall include a project-unique identifier and shall designate the interfacing entities (systems, configuration items, users, etc.) by name, number, version, and documentation references, as applicable. The identification shall state which entities have fixed interface characteristics (and therefore impose interface requirements on interfacing entities) and which are being developed or modified (thus having interface requirements imposed on them). One or more interface diagrams shall be provided to depict the interfaces.]

\subsection{[Project-Unique Identifier of Interface]\label{ref-011}}

[This paragraph (beginning with 3.3.2) shall identify a software item external interface by project-unique identifier, shall briefly identify the interfacing entities, and shall be divided into subparagraphs as needed to state the requirements imposed on the software item to achieve the interface. Interface characteristics of the other entities involved in the interface shall be stated as assumptions or as ``When [the entity not covered] does this, the software item shall ... ,'' not as requirements on the other entities. This paragraph may reference other documents (such as data dictionaries, standards for communication protocols, and standards for user interfaces) in place of stating the information here. The requirements shall include the following, as applicable, presented in any order suited to the requirements, and shall note any differences in these characteristics from the point of view of the interfacing entities (such as different expectations about the size, frequency, or other characteristics of data elements):

\begin{itemize}
\item Priority that the software items must assign the interface.

\item Requirements on the type of interface (such as real-time data transfer, storage-and-retrieval of data, etc.) to be implemented.

\item Required characteristics of individual data elements that the software item must provide, store, send, access, receive, etc., such as:

\begin{itemize}
\item Names/identifiers

\begin{itemize}
\item Project-unique identifier

\item Non-technical (natural-language) name

\item Standard data element name

\item Technical name (e.g., variable or field name in code or database)

\item Abbreviation or synonymous names

\end{itemize}
\item Data type (alphanumeric, integer, etc.)

\item Size and format (such as length and punctuation of a character string)

\item Units of measurement (such as meters, dollars, nanoseconds)

\item Range or enumeration of possible values (such as 0-99)

\item Accuracy (how correct) and precision (number of significant digits)

\item Priority, timing, frequency, volume, sequencing, and other constraints, such as whether the data element may be updated and whether business rules apply

\item Security and privacy constraints

\item Sources (setting/sending entities) and recipients (using/receiving entities)

\end{itemize}
\item Required characteristics of data element assemblies (records, messages, files, arrays, displays, reports, etc.) that the software item must provide, store, send, access, receive, etc., such as:

\begin{itemize}
\item Names/identifiers

\begin{itemize}
\item Project-unique identifier

\item Non-technical (natural-language) name

\item Technical name (e.g., variable or field name in code or database)

\item Abbreviation or synonymous names

\end{itemize}
\item Data elements in the assembly and their structure (number, order, grouping)

\item Medium (such as disk) and structure of data elements/assemblies on the medium

\item Visual and auditory characteristics of displays and other outputs (such as colors, layouts, fonts, icons, and other display elements, beeps, lights)

\item Relationships among assemblies, such as sorting/access characteristics

\item Priority, timing, frequency, volume, sequencing, and other constraints, such as whether the assembly may be updated and whether business rules apply

\item Security and privacy constraints

\item Sources (setting/sending entities) and recipients (using/receiving entities)

\end{itemize}
\item Required characteristics of communication methods that the software item must use for the interface, such as:

\begin{itemize}
\item Project-unique identifier(s)

\item Communication links/bands/frequencies/media \& their characteristics

\item Message formatting

\item Flow control (such as sequence numbering and bullet allocation)

\item Data transfer rate, whether periodic/aperiodic, and interval between transfers

\item Routing, addressing, and naming conventions

\item Transmission services, including priority and grade

\item Safety/security/privacy considerations, such as encryption, user authentication, compartmentalization, and auditing

\end{itemize}
\item Required characteristics of protocols the software item must use for the interface, such as:

\begin{itemize}
\item Project-unique identifier(s)

\item Priority/layer of the protocol

\item Packeting, including fragmentation and re-assembly, routing, and addressing

\item Legality checks, error control, and recovery procedures

\item Synchronization, including connection establishment, maintenance, termination

\item Status, identification, and any other reporting features

\end{itemize}
\item Other required characteristics, such as physical compatibility of the interfacing entities (dimensions, tolerances, loads, plug compatibility, etc.), voltages, etc.]

\end{itemize}
\section{Software Item Internal Interface Requirements\label{ref-012}}

[This paragraph shall specify the requirements, if any, imposed on interfaces internal to the software item. If all internal interfaces are left to the design, this fact shall be so stated. If such requirements are to be imposed, paragraph 3.3 of this product description provides a list of topics to be considered.]

\section{Software Item Internal Data Requirements\label{ref-013}}

[This paragraph shall specify the requirements, if any, imposed on data internal to the software item. Included shall be requirements, if any, on databases and data files to be included in the software item. If all decisions about internal data are left to the design, this fact shall be so stated. If such requirements are to be imposed, paragraphs 3.3.x (3rd bullet) and 3.3.x (4th bullet) of this product description provide a list of topics to be considered.]

\section{Adaptation Requirements\label{ref-014}}

[This paragraph shall specify the requirements, if any, concerning installation-dependent data to be provided by the software item (such as site-dependent latitude and longitude or site-dependent state tax codes) and operational parameters that the software item is required to use that may vary according to operational needs (such as parameters indicating operation-dependent targeting constants or data recording).]

\section{Safety Requirements\label{ref-015}}

[This paragraph shall specify the software item requirements, if any, concerned with preventing or minimizing unintended hazards to personnel, property, and the physical environment. Examples include safeguards the software item must provide to prevent inadvertent actions (such as accidentally issuing an ``auto pilot off'' command) and non-actions (such as failure to issue an intended ``auto pilot off'' command). This paragraph shall include the software item requirements, if any, regarding nuclear components of the system, including, as applicable, prevention of inadvertent detonation and compliance with nuclear safety rules.]

\section{Security and Privacy Requirements\label{ref-016}}

[This paragraph shall specify the software item requirements, if any, concerned with maintaining security and privacy. These requirements shall include, as applicable, the security/privacy environment in which the software item must operate, the type and degree of security or privacy to be provided, the security/privacy risks the software item must withstand, required safeguards to reduce those risks, the security/privacy policy that must be met, the security/privacy accountability the software item must provide, and the criteria that must be met for security/privacy certification/accreditation.]

\section{Software Item Environment Requirements\label{ref-017}}

[This paragraph shall specify the requirements, if any, regarding the environment in which the software item must operate. Examples include the computer hardware and operating system on which the software item must run. (Additional requirements concerning computer resources are given in the next paragraph.)]

\section{Computer Resource Requirements\label{ref-018}}

[This paragraph shall be divided into the following subparagraphs.]

\subsection{Computer Hardware Requirements\label{ref-019}}

[This paragraph shall specify the requirements, if any, regarding computer hardware that must be used by the hardware item. The requirements shall include, as applicable, number of each type of equipment, type, size, capacity, and other required characteristics of processors, memory, input/output devices, auxiliary storage, communications/network equipment, and other required equipment.]

\subsection{Computer Hardware Resource Utilization Requirements\label{ref-020}}

[This paragraph shall specify the requirements, if any, on the hardware item's computer hardware resource utilization, such as maximum allowable use of processor capacity, memory capacity, input/output device capacity, auxiliary storage capacity, and communications/network equipment capacity. The requirements (stated, for example, as percentages of the capacity of each computer hardware resource) shall include the conditions, if any, under which the resource utilization is to be measured.]

\subsection{Computer Software Requirements\label{ref-021}}

[This paragraph shall specify the requirements, if any, regarding computer software that must be used by, or incorporated into, the software item. Examples include operating systems, database management systems, communications/network software, utility software, input and equipment simulators, test software, and manufacturing software. The correct nomenclature, version, and documentation references of each such software item shall be provided.]

\subsection{Computer Communications Requirements\label{ref-022}}

[This paragraph shall specify the additional requirements, if any, concerning the computer communications that must be used by the software item. Examples include geographic locations to be linked; configuration and network topology; transmission techniques; data transfer rates; gateways; required system use times; type and volume of data to be transmitted/received; time boundaries for transmission/reception/response; peak volumes of data; and diagnostic features.]

\section{Software Quality Factors\label{ref-023}}

[This paragraph shall specify the software item requirements, if any concerned with software quality factors identified in the contract or derived from a higher level specification. Examples include quantitative requirements regarding software item functionality (the ability to perform all required functions), reliability (the ability to perform with correct, consistent results), maintainability (the ability to be easily corrected), availability (the ability to be accessed and operated when needed), flexibility (the ability to be easily adapted to changing requirements), portability (the ability to be easily modified for a new environment), reusability (the ability to be used in multiple applications), testability (the ability to be easily and thoroughly tested), usability (the ability to be easily learned and used), and other attributes.]

\section{Design and Implementation Constraints\label{ref-024}}

[This paragraph shall specify the requirements, if any, that constrain the design and implementation of the software item. These requirements may be specified by reference to appropriate commercial standards and specifications. Examples include requirements concerning:

\begin{itemize}
\item Use of a particular software/hardware item architecture or requirements on the architecture, such as required databases or other software units; use of standard, or existing components; or use of furnished property (equipment, information, or software)

\item Use of particular design or implementation standards; use of particular data standards; use of a particular programming language

\item Flexibility and expandability that must be provided to support anticipated areas of growth or changes in technology, threat, or mission.]

\end{itemize}
\section{Personnel-Related Requirements\label{ref-025}}

[This paragraph shall specify the software/hardware item requirements, if any, included to accommodate the number, skill levels, duty cycles, training needs, or other information about the personnel who will use or support the software item. Examples include requirements for number of simultaneous users and for built-in help or training features. Also included shall be the human factors engineering requirements, if any imposed on the software item. These requirements shall include, as applicable, considerations for the capabilities and limitations of humans; foreseeable human errors under both normal and extreme conditions; and specific areas where the effects of human error would be particularly serious. Examples include requirements for color and duration of error messages, physical placement of critical indicators or keys, and use of auditory signals.]

\section{Training-Related Requirements\label{ref-026}}

[This paragraph shall specify the software/hardware item requirements, if any, pertaining to training. Examples include training software to be included in the software/hardware item.]

\section{Logistics-Related Requirements\label{ref-027}}

[This paragraph shall specify the software item requirements, if any, concerned with logistics considerations. These considerations may include: system maintenance, software support, system transportation modes, supply-system requirements, impact on existing facilities, and impact on existing equipment.] 

\section{Packaging Requirements\label{ref-028}}

[This section shall specify the requirements, if any, for packaging, labeling, and handling the software items for delivery (for example, delivery on magnetic tape labeled and packaged in a certain way). Applicable commercial specifications and standards may be referenced if appropriate.]

\section{Other Requirements\label{ref-029}}

[This paragraphs shall specify additional software/hardware item requirements, if any, not covered in the previous paragraphs.] 
